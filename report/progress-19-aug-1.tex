% !TeX document-id = {e3e9adc9-0b17-4199-b894-b6e314b8cf66}
% !TeX spellcheck = en_US
% !BIB program = biber 
\documentclass{article}

%% Encoding
\usepackage[T1]{fontenc}
\usepackage[utf8]{inputenc}

%% Fonts
% Math fonts (fourier) with utopia (erewhon) text fonts
\usepackage{fourier, erewhon}

%% Setup
% This package contains logos
%\usepackage[autoload]{adn}

%\setlogos[
%\textbf{Laboratório de Computação Reconfigurável}\\%[5pt]
%\uppercase{Instituto de Ciências Matemáticas e de Computação - USP}\\%[-7pt]
%]%
%{IC3D}%
%{UNICAMP}

%% Transform section references
\makeatletter
\renewcommand*{\p@section}{\S\,}
\renewcommand*{\p@subsection}{\S\,}
\makeatother

%% Shorthands
\usepackage{xspace}
\makeatletter
\DeclareRobustCommand\onedot{\futurelet\@let@token\@onedot}
\def\@onedot{\ifx\@let@token.\else.\null\fi\xspace}

\def\eg{e.g\onedot} \def\Eg{E.g\onedot}
\def\ie{i.e\onedot} \def\Ie{I.e\onedot}
\def\cf{cf\onedot} \def\Cf{Cf\onedot}
\def\etc{etc\onedot} \def\vs{vs\onedot}
\def\wrt{w.r.t\onedot} \def\dof{d.o.f\onedot}
\def\etal{et al\onedot}
\makeatother

%%%
% Other packages start here (see the examples below)
%%

%% Figues
\usepackage{graphicx}
\graphicspath{{./images-2/}}


% References
% Use this section to embed your bibliography
% Instead of having a separate file, just place the bibtex entries here
\usepackage{filecontents}% create files
\begin{filecontents}{\jobname.bib}
  @book{FP,
    title={Computer Vision A Modern Approach},
    author={David A. Forsyth, Jean Ponce},
    year={2012},
    publisher={Pearson}
  }  
  @book{SZ,
    title={Computer Vision: Algorithms and Applications},
    author={Richard Szeliski},
    year={2010},
    publisher={Springer}
  }
\end{filecontents}
% Include bibliography file
\usepackage[
backend=biber, 
style=ieee, 
natbib=true,
]{biblatex}
\addbibresource{\jobname.bib}


%% Math
\usepackage{amsmath}


%% Enumerate
\usepackage{enumitem}


\begin{document}
\title{Algoritmo Genético para Planejadores de Rotas \\
	\large Relatório de Progresso II}
\author{Gustavo de Moura Souza\thanks{Número USP 9762981, gustavo.moura.souza@usp.br}}

\maketitle

\section{Indivíduo}
\subsection{Codificação}
Seja \(t\) o tamanho do horizonte de planejamento (quantidade de waypoints que deseja-se computar), o DNA do indivíduo é definido por:
\[dna = [ gene_1, gene_2, … gene_t ] \]

\(gene_i = [ a, e ]\)\hfill para \(i = 1, … t\)


onde: \(a\) é o ângulo, e \(e\) a aceleração


\subsection{Decodificação}
seja \(x\) o conjunto de controladores do drone, para \(i = 1, … t\):

\[x = [ (px_1, py_1, v_1, al_1)_1, … (px_t, py_t, v_t, al_t)_t]\]


\(px_i\) = : Posição do VANT no eixo x
 
\(py_i\) = : Posição do VANT no eixo y

\(v_i\)  = : Velocidade do VANT na horizontal

\(al_i\) = : ângulo (direção) do VANT na horizontal



\section{Algoritmo Genético}
\subsection{Genesis - Criação da População}
Gera uma população de \(S\) indivíduos. O gene de cada indivíduo é atribuído de acordo com uma função e segue uma distribuição uniforme:

\(a\) = uniformemente distribuído entre \(0.5\) e \(10\)

\(e\) = uniformemente distribuído entre \(0.5\) e \(2\pi\)

\subsection{Evaluation - Computar o Valor do Fitness}
Função de fitness sendo utilizada:
\[fitness = f_{pouso\_b} + f_{pouso\_p} + f_{pouso\_voo\_n} + f_{viol} + f_{curvas}\]

onde:

\begin{itemize}
	\item \(f_{pouso\_b}\) : pouso em região bonificadora
	\[f_{pouso\_b}(x_K, mapa) = - Cb \cdot \sum(Pr(x \in Z) )\]
	Custo de pousar em região bonificadora \(Cb\) vezes a somatória da probabilidade de pousar em cada uma das regiões bonificadoras
	
	\item \(f_{pouso\_p}\) : pouso em região penalizadora
	
	igual a \(f_{pouso\_b}\), porém substituíndo \(-Cb\) por \(+Cp\) (custo de pousar em região penalizadora)
	
	\item \(f_{pouso\_voo\_n}\) : pouso ou voo sobre região não-navegável
	\[f_{pouso\_voo\_n} = Cn \cdot max(0 , calc)\]
	\[calc = 1 - delta - \sum(Pr(x \notin Z))\]
	
	Um menos delta menos somatória da somatória das probabilidades de cada um dos \(x\) não pertencer à cada uma das regiões de \(Zn\)  (loop for duplo, para cada um do controlador \(x\) a cada uma das regiões \(Zn\)), onde \(Zn\) é a região definida como não-navegável.
	
	\item \(f_{viol}\) = realiza uma comparação de segurança
	
	Se a aeronave tem velocidade final maior do que o seu valor mínimo, não ocorre de fato um pouso. Dessa maneira, a Equação \(f_{viol}\) evita rotas em que o VANT não consegue pousar, mesmo que atinja uma região bonificadora.
	
	\item \(f_{viol}=Cb\) , se \(vk - vminimo > 0\); 
	\(0\), caso contrário
	
	note que nesse caso o \(Cb\) está positivo, aumentando muito o valor do fitness. Lembrando que o objetivo é minizar o fitness, ou seja, quanto menor o fitness, melhor.
	

\end{itemize}

\subsubsection{Probability}

	\(Pr(x \in Z) \) e \(Pr(x \notin Z) \), Probabilidade de x pertencer ou não à uma região Z:
	
Seja \(x\) o elemento decodificado (definido no ínicio) utiliza-se a posição cartesiana \(px\) e \(py\) para definir um ponto. Ponto este representando a posição do drone no espaço. Seja uma região composta por 4 ou mais pontos geográficos (que são convertidos para cartesiano para efeito de cálculo) definindo assim uma área.

A probabilidade é:

AAAAAAAAAAAAAAAAAAAAAAAAAAAAAAAAAAAAAAAAAAAA

\vspace{5mm} %5mm vertical space
\textit{\textbf{OBS 1}}: 

Segundo a tese do Jesimar, a função de fitness completa seria:
\[fitness = f_{pouso\_b} + f_{pouso\_p} + f_{pouso\_voo\_n} + f_{viol} + f_{curvas} + f_{dist} + f_{bat}\]

onde \(f_{curvas}, f_{dist}, f_{bat}\) eu não entendi como implementar, portanto foi definido como constante \(0\).


\vspace{5mm} %5mm vertical space

\textit{\textbf{OBS 2}}:

Em alguns momentos é utilizado o valor \(K\). Segundo a tese, \(K\) é o instante de tempo em que o drone sofre um acidente e entra em modo de recalcular a rota.
Como essa situação não é prevista na implementação do meu algorítimo, \(K\)  é igual ao valor de \(t\). Ou seja, a posição final do drone.
Esse valor é utilizado para os cálculos de:  \(f_{pouso\_b}\), \(f_{pouso\_p}\) e \(f_{viol}\) 



\subsection{Selection}
O indivíduo com melhor (menor) valor de fitness é selecionado, chamado de “melhor de todos”


\subsection{Crossover}
Seguindo o modelo de o melhor cruza com todos. É realizada a média entre os valores dos pais, para determinar os valores do filho.
Segue a expressão para cada um dos genes do DNA:
\[a_i^{filho} = ( a_i^{melhor} + a_i^{indivíduo}) / 2\]
\[e_i^{filho} = ( e_i^{melhor} + e_i^{indivíduo}) / 2\]
\[gene_i^{filho} = ( a_i^{filho}, e_i^{filho})\]


\subsection{Mutation}
O passo de mutação é realizado sobre cada um dos genes dos novos indivíduos, seguindo a expressão:
\[a = a + taxa\_mutação \cdot var\]

onde \(var = 1 ou -1\), selecionado aleatoriamente

ou seja, a mutação soma ou subtrai um pequeno valor constante definido inicialmente

verificar se acrescimo ou descrescimo, excede o range do limite


\subsection{Extintion}
A população que já procriou é extinguida, exceto pelo melhor indíviduo, onde ele será perpetuado para a próxima geração. A nova geração de t-1 indivíduos com o melhor de todos segue a vida para a próxima época.

Por motivos de segurança, é guardada a linhagem dos “melhor de todos”, sempre mantendo um histórico.


\printbibliography

\end{document}